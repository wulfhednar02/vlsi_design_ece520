\documentclass[12pt, letterpaper]{article}

\usepackage[margin=1.0in]{geometry} % less huge margins

\usepackage{siunitx} % for describing units
\newcommand{\micron}{\micro\meter}

% nice diagonal fractions with SI
\usepackage{xfrac}
\sisetup{per-mode=fraction, fraction-function=\sfrac}

\usepackage{lmodern} % get rid of font size issues

\title{1-Wire Transceiver Project Phase 1}
\author{Charles Helmich \and Ivan Bow \and Don Owen}
\date{February 25, 2019}

% macros for common parameters
\newcommand{\Vm}{$V_m$ } % switching voltage
\newcommand{\Wn}{$W_n$ } % NMOS width
\newcommand{\Wp}{$W_p$ } % PMOS width
\newcommand{\RXp}{$\overline{\mbox{RX}}$ } % inverted RX

\begin{document}

\maketitle

\tableofcontents

\section{Description of Design}
\subsection{Functional Description}
The operation of the one-wire channel with a transceiver on either end will produce 3 possible voltages on the channel: 0, 5, and 2.5 volts.
When both transceivers are sending the same value, the voltage on the channel will be that value, either 0 or 5 volts.
When the transceivers are are sending opposite values, the value on the channel will be halfway between the two TX values because the external resistors will form a voltage divider.
Based on these possible values on the channel it is straightforward to produce a logic table that unambiguously decodes what the received value at a transceiver is.

\begin{table}[htb]
    \centering
    \begin{tabular}{c c c}
        Channel & TX (local) & RX (remote tx) \\
        \hline
        0   & 0 & 0 \\
        2.5 & 0 & 5 \\
        2.5 & 5 & 0 \\
        5   & 5 & 5 \\
        \hline
    \end{tabular}
    \caption{Voltage logic table}
\end{table}

In order to differentiate between the 3 possible channel voltages we chose to design a pair of imbalanced inverters that have mid-point voltages that are either 1.25 or 3.75 volts, corresponding to halfway between 0 and 2.5 volts and halfway between 2.5 and 5 volts, respectively.
The inverter with a low \Vm has a much stronger nFET than pFET and the high \Vm is the opposite.
\par
The n-strong inverter will be on when the channel voltage is 2.5 or 5 volts while the p-strong inverter will be on when the channel voltage is 2.5 or 0 volts.
The next stage of the receiver uses the local TX signal to control a mux that selects either the n-strong or p-strong output.
When the local TX is high, it selects the p-strong output.
The p-strong output will be logically 1 when the channel is low (2.5V) and 0 when the channel is high (5V), corresponding logically to \RXp.
When the local TX is low, the n-strong output will be logically 1 when the channel is low (0V) and 0 when the channel is high (2.5V), corresponding logically to \RXp.
\par
We note that because we chose to design the mux with a single nFET and pFET the value passes through is degraded and not driven to the full 0-5V range.
This does not have a significant impact on the function of the receiver because the later stages both invert and buffer the \RXp signal, giving it a full 0-5V swing.
The final stage in the receiver is an inverter chain that rectifies the \RXp signal into an RX signal and buffers the output to drive non-trivial load capacitances.

\subsection{Logic Sizing}

\subsubsection{Inverters}
With the $\mu_0$, $V_{th}$, and $T_{ox}$, parameters taken from provided model library, we calculated the theoretical width ratio of the pFET to the nFET for the for the 3 different \Vm values used in our design according to the following formula:

\[
\frac{K'_n}{2} \frac{W_n}{L} (V_m - V_{tn}) ^ 2 =
\frac{K'_p}{2} \frac{W_p}{L} (V_{DD} - V_m - |V_{tp}|) ^ 2
\]

Based on that formula, we calculated the width ratios for imbalanced and balanced inverters that gave us the \Vm values we needed for our design.
These are shown in the table below.

\begin{table}[htb]
    \centering
    \begin{tabular}{c c c}
        \Vm & \Wp & \Wn \\
        \hline
        1.25V & 1     & 9.27 \\
        2.5V  & 3.125 & 1 \\
        3.25V & 194.5 & 1 \\
        \hline
    \end{tabular}
    \caption{Switching voltage ratios (first order approximation)}
\end{table}

Because the above calculations only take into account the first-order effects specified in the model library, we also instantiated the inverter designs into a SPICE simulation and experimentally measured the width ratios that gave us the required \Vm values.
These are shown in the table below.
These (or close approximations) are the width ratios that we used in the designs for the project.

\begin{table}[htb]
    \centering
    \begin{tabular}{c c c}
        \Vm & \Wp & \Wn \\
        \hline
        1.25V & 1     & 4.126 \\
        2.5V  & 2.258 & 1 \\
        3.25V & 44.25 & 1 \\
        \hline
    \end{tabular}
    \caption{Switching voltage ratios (simulated)}
\end{table}

\subsubsection{2-Transistor-Mux}
The 2-Transistor Mux (2TMux) looks like a 1X balanced inverter in design, except with the supply of the nFET and pFET wired to the 2 signals to be multiplexed.
When driven high, the gate selects the n supply signal to pass through and vice versa when low.
This design will never produce a full 0-VDD voltage swing, but does allow us to mux the output of our imbalanced inverters with only 2 transistors and the TX signal, avoiding the cost of inverting the TX and creating a full transmission gate mux.
The output of the mux is fed into a balanced inverter with a \Vm of 2.5; this inverter will produce an almost rail to rail swing because neither transistor will ever be fully off and further buffering stages will restore the rail-to-rail swing.

\subsubsection{Output Buffer}
Based on the parallel plate capacitor model we calculated that the gate capacitance for the process was

\[
C_{gate} = C_{ox} = \frac{\mathcal{E}_{ox}}{T_{ox}} =
\frac{3.9 \times \num{8.854e-12}}{\num{1.38e-8}} \times
\frac{\SI{1}{\meter\squared}}{\SI{1e12}{\micron\squared}} =
\SI{2.48}{\femto\farad\per\micron\squared}
\]

With a balanced 1X inverter design, the resulting capacitance was then

\[
\SI{2.48}{\femto\farad\per\micron\squared} \times L \times (W_n + W_p) =
\SI{2.48}{\femto\farad\per\micron\squared} \times \SI{0.6}{\micron} \times (\SI{3}{\micron} + \SI{6.75}{\micron}) =
\SI{14.5}{\femto\farad}
\]

With an assumed output capacitance load of \SI{20}{\pico\farad} the resulting fanout from the \RXp signal to the load was approximately 1400.
The optimal number of optimal-fanout stages is then ln(1400) = 7.2.
Because we needed to invert the \RXp signal, we chose to use an odd and therefore inverting 7-stage output inverter chain with a ratio of 3x fanout per stage (for simplicity of design) to buffer the \RXp signal to the assumed load capacitance on RX.

\section{Waveforms of Sub-designs}
\subsection{Imbalanced Inverters}
\subsection{Pass-Mux}
\subsection{Balanced Inverters and Output Buffer}

\section{Waveforms and Measurements of full Transceiver}
\subsection{Propagation Delay}
\subsection{Rise Time}
\subsection{Fall Time}

\end{document}
